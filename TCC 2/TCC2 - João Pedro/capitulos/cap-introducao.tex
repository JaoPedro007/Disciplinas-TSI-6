%%%% CAPÍTULO 1 - INTRODUÇÃO
%%
%% Deve apresentar uma visão global da pesquisa, incluindo: breve histórico, importância e justificativa da escolha do tema,
%% delimitações do assunto, formulação de hipóteses e objetivos da pesquisa e estrutura do trabalho.

%% Título e rótulo de capítulo (rótulos não devem conter caracteres especiais, acentuados ou cedilha)
\chapter{Introdução}\label{cap:introducao}

Nos últimos anos, avanços consideráveis na pesquisa e desenvolvimento de tecnologia quântica têm levantado questões sobre a segurança dos sistemas de criptografia convencional. O poder de processamento das informações dos computadores quânticos, por realizarem operações em paralelo e explorarem os princípios da superposição e emaranhamento, representa uma ameaça significativa à criptografia convencional utilizada atualmente para proteger dados sensíveis em comunicações digitais. Enquanto a criptografia convencional depende da complexidade computacional para proteger os dados, os computadores quânticos podem potencialmente quebrar esses algoritmos em um tempo muito menor, comprometendo a confidencialidade e integridade das comunicações digitais \cite{mitra2017quantum}. Computadores quânticos atualmente encontram-se em estágios iniciais de desenvolvimento e aplicação. Empresas e instituições de pesquisa ao redor do mundo estão trabalhando ativamente para construir e aprimorar esses sistemas. Alguns exemplos notáveis incluem os esforços da IBM, Google, Microsoft, Intel, Rigetti Computing, entre outros, bem como instituições acadêmicas e laboratórios de pesquisa governamentais \cite{nandhini2022extensive}. Embora os computadores quânticos ainda estejam longe de serem tão poderosos e amplamente disponíveis como os computadores clássicos, há avanços significativos sendo feitos nessa área. Esses avanços têm implicações profundas para vários campos, incluindo a criptografia.

Diante do cenário de avanços na pesquisa e desenvolvimento de tecnologia quântica e das preocupações crescentes com a segurança dos sistemas de criptografia convencional, surge um problema de destaque: a estratégia conhecida como ``armazenar agora, decifrar depois'' (\textit{store now, decrypt later}). Essa abordagem baseia-se na aquisição e armazenamento a longo prazo de dados criptografados atualmente ilegíveis, aguardando possíveis avanços na tecnologia de decifragem que permitiriam sua leitura no futuro, em uma data hipotética referida como Y2Q (uma alusão ao Y2K \textit{problem}, ou \textit{bug} do milênio) \cite{argetsinger2024promise}. Esta estratégia, que espera explorar futuras quebras na criptografia para acessar informações previamente inacessíveis, é o problema principal abordado nesta proposta de trabalho. A partir dessa problemática, propõe-se um estudo sobre os métodos atualmente disponíveis para mitigar os riscos associados à estratégia ``armazenar agora, decifrar depois". Em suma, os métodos criptográficos atuais enfatizam a adoção de algoritmos de  ``criptografia pós-quântica'' (CPQ), oferecendo uma abordagem resistente à ameaça iminente dos avanços na computação quântica.

A segurança das comunicações digitais é essencial para a proteção da privacidade, integridade e autenticidade das informações. Diante da chegada dos computadores quânticos, a busca por soluções criptográficas pós-quânticas torna-se uma necessidade, visando garantir a segurança dos dados em um ambiente de ameaças em constante evolução.

O principal desafio deste projeto consiste na identificação e apresentação das novas técnicas criptográficas capazes de resistir aos potenciais ataques quânticos, e na compreensão dos fundamentos teóricos subjacentes a essas técnicas. A complexidade dos algoritmos de criptografia pós-quântica apresenta um desafio significativo, exigindo um entendimento de conceitos matemáticos e computacionais.

Este trabalho pretende contribuir para uma maior compreensão dos desafios e soluções na área da segurança da informação em um contexto pós-quântico. Ao identificar e apresentar os algoritmos de criptografia pós-quântica existentes, espera-se fornecer uma conscientização em relação à importância da proteção das comunicações digitais em um ambiente de ameaças em constante evolução.

\section{Objetivos}\label{sec:objetivos}
A seguir são apresentados os objetivos geral e específicos que regem esta proposta.

\subsection{Objetivo Geral}
\label{subsec:objgeral}

O objetivo geral desta proposta de TCC é apresentar um estudo sobre as soluções de criptografia pós-quântica disponíveis para assegurar a segurança das comunicações digitais frente aos desafios impostos pela computação quântica. Este estudo visa o entendimento das tecnologias criptográficas que são projetadas para serem resilientes contra os métodos de quebra que computadores quânticos podem oferecer, uma vez que se tornem operacionais. Para tanto, o trabalho focará na revisão e apresentação dos algoritmos de criptografia pós-quântica mais relevantes e nos princípios que garantem sua eficiência. O trabalho pretende abordar algoritmos já estabelecidos e também, possivelmente, aqueles ainda em desenvolvimento, proporcionando uma visão ampla do estado atual da criptografia pós-quântica e suas potenciais aplicações práticas para proteger informações críticas em um futuro próximo.


\subsection{Objetivos Específicos}
\label{subsec:objespc}

Os objetivos específicos desta proposta de TCC incluem:
\begin{itemize}
	\item Identificar e catalogar os principais algoritmos de criptografia pós-quântica que estão atualmente em uso ou em fase de desenvolvimento, descrevendo suas bases teóricas, características e contextos de aplicação.
	\item Identificar as principais técnicas empregadas pelos algoritmos de criptografia pós-quântica.
	\item Apresentar uma comparação sobre os algoritmos de criptografia pós-quântica entre si em termos de complexidade computacional e praticabilidade, utilizando critérios de avaliação para determinar suas vantagens e limitações.
	\item Elaborar recomendações para desenvolvedores web, tendo em visto as deficiências nos atuais algoritmos de criptografia, frente a potenciais aplicações práticas de computadores quânticos.
\end{itemize}

\section{JUSTIFICATIVA}
\label{sec:justificativa}

A relevância do problema abordado nesta proposta de trabalho reside na importância crescente da segurança da informação em um cenário onde avanços na computação quântica ameaçam comprometer os sistemas de criptografia convencionais. A possibilidade iminente de que computadores quânticos possam quebrar algoritmos de criptografia existentes representa uma ameaça significativa à confidencialidade e integridade das comunicações digitais atuais.

Dentre as contribuições deste trabalho, espera-se primeiramente, identificar e apresentar as novas técnicas criptográficas capazes de resistir aos potenciais ataques quânticos, proporcionando uma compreensão dos desafios e soluções na área da segurança da informação em um contexto pós-quântico. Além disso, o estudo dessas técnicas pode proporcionar informações importantes para o desenvolvimento de sistemas web seguros.

Quanto ao estágio de desenvolvimento dos conhecimentos referentes ao tema, embora os computadores quânticos ainda estejam em estágios iniciais de desenvolvimento e aplicação, os avanços significativos na pesquisa e desenvolvimento de tecnologia quântica nos últimos anos destacam a importância de antecipar e preparar-se para os desafios que esses avanços podem trazer para a segurança da informação.

Por fim, este trabalho tem o potencial de sugerir recomendações para desenvolvedores web, considerando as deficiências nos atuais algoritmos de criptografia frente às potenciais aplicações práticas de computadores quânticos. Ao identificar e descrever os principais algoritmos de criptografia pós-quântica, bem como suas características e contextos de aplicação, pretende-se oferecer orientações práticas para a implementação de medidas de segurança mais robustas e resilientes no ambiente digital.

Assim, considerando a relevância do problema, as contribuições esperadas, o estágio de desenvolvimento dos conhecimentos referentes ao tema e a possibilidade de sugerir recomendações práticas, justifica-se a realização deste trabalho de TCC.

\section{Estrutura do trabalho}\label{sec:estruturaTrabalho}

Este trabalho está estruturado da seguinte forma...

%Um texto curto apresentando o capítulo.
%
%\caixa{Atenção}{Para utilizar esse template é obrigatória a leitura do conteúdo do arquivo \texttt{readme.md}, que está neste projeto!}
%
%\section{Considerações iniciais}\label{secconsideracoes:Iniciais}
%As considerações iniciais compõem um texto curto e geral apresentando uma visão geral e sucinta do assunto principal relacionado ao trabalho e a inserção do objeto de pesquisa nesse assunto %\cite{Moore:2000:CMC:333067.333074}.
%
%Em relação ao assunto, o apresentado nesta seção pode estar relacionado a trabalhos de outros autores ou ao assunto que fornece a fundamentação (motivação) para o trabalho a ser desenvolvido. Se o assunto está relacionado a trabalhos de outros autores, a contribuição do trabalho é definida em relação ao que já foi pesquisado nesse assunto. Se o assunto será utilizado para embasamento do que será proposto, explicitar como o trabalho se insere nesse assunto. A contribuição pode, ainda, estar relacionada a uma necessidade de mercado ou a uma oportunidade decorrente de algum problema real para o qual se pretender propor uma solução. Nesse caso, o assunto fornece um contexto teórico de suporte para o problema e/ou a solução.
%
%O importante nesta seção é deixar claro do que se trata o trabalho (assunto ou tema), identificar o objeto de pesquisa, como será encaminhada a solução (procedimento metodológico, tecnologias, ferramentas utilizadas) e o que se pretende ao final do trabalho, sem explicitar a solução e os resultados.
%
%\caixa{Atenção}{As seções a seguir são sugestões, converse com o seu orientador para ver quais seções devem ter em seu trabalho!}
%
%\section{Objetivos}\label{sec:objetivos}
%
%Um texto curto\footnote{Teste de nota de rodapé 1.} apresentando a seção.
%
%\subsection{Objetivo geral}\label{subsec:objetivoGeral}
%
%O objetivo geral se refere ao resultado do trabalho realizado, enfatizando o que esse trabalho deixa para a comunidade acadêmica, para a sociedade e/ou para o ambiente profissional. Deve ser apresentado de forma a abranger o resultado principal do teste.
%
%O objetivo geral e os específicos devem iniciar com verbo. Sugere-se que o objetivo geral contenha no máximo 3 (três) linhas, conforme exemplo abaixo:
%
%Desenvolver um protótipo de um sistema de software para determinar a capacidade produtiva de pequenas empresas com base em estudos de cronoanálise industrial para pequenas empresas com produção em série.
%
%\subsection{Objetivos específicos (opcional)}\label{subsec:objetivosEspecificos}
%
%Os objetivos específicos são opcionais, ou seja, somente devem ser apresentados se caracterizarem resultados parciais gerados a partir do objetivo geral, os quais sejam considerados úteis para a comunidade acadêmica, para a sociedade ou para o ambiente profissional. Uma observação importante é que os resultados sejam passíveis de comprovação, ou seja, se o objetivo for: “Oferecer agilidade e confiabilidade aos processos gerenciais da empresa”, significa que o trabalho deverá realizar testes com relação a esses atributos, cujos resultados deverão ser apresentados nas discussões do trabalho.
%
%Destaca-se que os objetivos específicos não incluem as etapas do processo de desenvolvimento de software (realizar a modelagem, a análise, o projeto...) ou outras atividades necessárias para alcançar o objetivo geral, como, estudar as tecnologias necessárias para modelagem e implementação do sistema. Dentre as exceções estão a realização de estudos, procedimentos, métodos e técnicas considerados inéditos e de relevância para outros trabalhos a serem realizados na mesma área. Contudo, o resultado deste estudo deve ser documentado de forma que seja conhecimento disponibilizado para quem lê o trabalho.
%
%\section{Justificativa}\label{sec:justificativa}
%
%Justificar o objeto de pesquisa (o que será feito) e a forma de resolução do problema (como fazer). A forma de resolução pode estar centrada no método, nas tecnologias, no uso de conceitos (fundamentação teórica).
%
%A Justificativa explicita porque desenvolver o referido trabalho, como o mesmo se insere no contexto de pesquisa, de produção científica. Pode incluir o porquê utilizar as tecnologias e ferramentas indicadas, a contribuição em termos de inovação ou mesmo de aprendizado.
%
%O trabalho não precisa ser justificado em decorrência de ser inovador ou por ter gerado uma significativa contribuição ao conhecimento na área em que o mesmo se insere. Pode referir-se simplesmente à aplicabilidade de conhecimentos adquiridos durante o curso. Sendo assim, a justificativa não deve ser elaborada considerando um mercado a ser atingido e sim com relação ao uso de tecnologias aprendidas e/ou estudadas, o conhecimento e aprendizado do aluno e a aplicabilidade do trabalho desenvolvido.
%
%\section{Estrutura do trabalho}\label{sec:estruturaTrabalho}
%
%A estrutura do trabalho contém uma relação dos capítulos e uma descrição sucinta do que cada um deles contém. Esta seção fornece uma visão geral do trabalho no sentido da sua estrutura em capítulos\footnote{Teste de nota de rodapé 2.}.
%
%\caixa{Atenção}{O OverLeaf está demorando muito para compilar o modelo com o Capítulo de Exemplos, que explica como usar o LaTeX. Assim, esse capítulo foi removido (está comentado para não compilar), mas há um arquivo chamado \texttt{exemploPDF.pdf}, na raiz do projeto, que contém esse capítulo de exemplos!}
