%%%% CAPÍTULO 2 - REVISÃO DA LITERATURA (OU REVISÃO BIBLIOGRÁFICA, ESTADO DA ARTE, ESTADO DO CONHECIMENTO)
%%
%% O autor deve registrar seu conhecimento sobre a literatura básica do assunto, discutindo e comentando a informação já publicada.
%% A revisão deve ser apresentada, preferencialmente, em ordem cronológica e por blocos de assunto, procurando mostrar a evolução do tema.
%% Título e rótulo de capítulo (rótulos não devem conter caracteres especiais, acentuados ou cedilha)
\chapter{Referencial te\'orico}\label{cap:referencialTeorico}


A computação quântica é uma área da ciência da computação e da física que utiliza os princípios da mecânica quântica para realizar operações. Diferentemente dos computadores clássicos, que usam bits binários (0 ou 1) para processar informações, os computadores quânticos utilizam qubits, que podem existir em superposição de estados. Isso significa que um qubit pode ser 0, 1 ou ambos ao mesmo tempo, permitindo que os computadores quânticos processem uma quantidade exponencialmente maior de informações em paralelo.

\section{História da Computação Quântica}

A história da computação quântica começa na década de 1980, com a contribuição de diversos cientistas que perceberam o potencial dos sistemas quânticos para realizar cálculos. Alguns dos principais pioneiros desse campo incluem Richard Feynman, David Deutsch, Peter Shor e Lov Grover.

Em 1981, durante uma conferência sobre física da computação no MIT (Massachusetts Institute of Technology), Richard Feynman propôs a ideia de que os computadores clássicos não eram capazes de simular sistemas quânticos de forma eficiente. Ele sugeriu a necessidade de um novo tipo de computador que fosse baseado em princípios quânticos \cite{feynman2018simulating}. Essa ideia de Feynman foi fundamental para o desenvolvimento inicial da computação quântica, pois abriu caminho para a exploração de conceitos e tecnologias que poderiam aproveitar o poder dos fenômenos quânticos.

Em 1985, o físico britânico David Deutsch formalizou a ideia de um computador quântico universal. Ele introduziu o conceito de uma máquina de Turing quântica, demonstrando que um computador quântico poderia simular qualquer sistema físico, inclusive computadores clássicos \cite{deutsch1985quantum}. Deutsch começou a explorar algoritmos quânticos que poderiam ser mais eficientes do que os clássicos, estabelecendo as bases teóricas para futuros desenvolvimentos na área.

Em 1994, Peter Shor desenvolveu um algoritmo quântico que revolucionou o campo da criptografia. O algoritmo de Shor é capaz de fatorar grandes números inteiros em tempo polinomial, o que representa uma ameaça à segurança de muitos sistemas criptográficos baseados em fatoração, como o RSA \cite{shor1994algorithms}. Essa descoberta destacou o enorme potencial dos computadores quânticos para resolver problemas que são praticamente insolúveis para computadores clássicos, mostrando uma das aplicações mais promissoras da computação quântica.

Em 1996, Lov Grover apresentou um algoritmo quântico para pesquisa em banco de dados, que é quadraticamente mais rápido do que qualquer algoritmo clássico equivalente \cite{grover1996fast}. O algoritmo de Grover é especialmente relevante para problemas de busca não estruturada, oferecendo melhorias significativas na eficiência e exemplificando como a computação quântica pode ser aplicada para resolver problemas complexos de forma mais rápida e eficaz.

\subsection{Comparação com a computação clássica}

A comparação entre a computação quântica e a computação clássica revela diferenças fundamentais na forma como essas duas abordagens processam informações e resolvem problemas. A compreensão dessas diferenças é essencial para compreender o potencial dos computadores quânticos em relação aos computadores clássicos.

A diferença mais básica entre computadores quânticos e clássicos reside na forma como eles representam e processam informações. Como já mencionado anteriormente, os computadores clássicos utilizam bits, que são unidades de informação binária representadas por 0 ou 1. Cada bit pode estar em um desses dois estados, e a computação clássica se baseia na manipulação de grandes conjuntos de bits para realizar cálculos.

Por outro lado, os computadores quânticos usam qubits (bits quânticos). Um qubit pode estar em um estado de 0, 1, ou qualquer superposição de ambos os estados ao mesmo tempo, graças ao fenômeno quântico da superposição. Essa capacidade de estar em múltiplos estados simultaneamente permite que os computadores quânticos processem uma quantidade exponencialmente maior de informações em paralelo, o que os torna potencialmente muito mais poderosos para certas tarefas.

Devido à capacidade dos computadores quânticos de realizar operações em muitos estados simultaneamente, eles são especialmente adequados para resolver problemas que envolvem um grande espaço de possibilidades. Algoritmos quânticos, como o de Shor para fatoração de números inteiros e o de Grover para busca em bases de dados, demonstram como os computadores quânticos podem superar os limites dos algoritmos clássicos em termos de velocidade de execução e eficiência.

Os computadores clássicos são altamente eficientes para a maioria das tarefas diárias e continuam a ser a melhor escolha para muitas aplicações. No entanto, para problemas específicos que envolvem grandes espaços de solução ou fenômenos quânticos intrinsecamente complexos, os computadores quânticos oferecem vantagens significativas.

Apesar de suas capacidades, os computadores quânticos enfrentam desafios significativos em termos de correção de erros e estabilidade. Os qubits são altamente sensíveis a interferências externas, o que pode levar à decorrência e erros nos cálculos. Em contraste, os computadores clássicos são mais estáveis e possuem técnicas bem estabelecidas de correção de erros. Desenvolver formas de mitigar os erros quânticos é um dos principais desafios na construção de computadores quânticos práticos e de larga escala.

Os computadores quânticos ainda estão em fases iniciais de desenvolvimento, com protótipos atualmente limitados a algumas dezenas ou centenas de qubits. No entanto, a percepção de que eles são apenas um projeto científico foi superada; agora, eles são um projeto de engenharia em desenvolvimento. Não resta dúvida de que esses sistemas não apenas funcionarão, mas também se tornarão viáveis comercialmente e práticos no futuro \cite{Callan2024}. À medida que a tecnologia avança, espera-se que os computadores quânticos revolucionem campos como a criptografia, a simulação de materiais e a otimização complexa, oferecendo soluções para problemas que são atualmente intratáveis com computadores clássicos.

\subsection{Fundamentos da Computação Quântica}

A computação quântica explora fenômenos quânticos únicos, como a superposição e o emaranhamento, que não têm equivalentes na computação clássica. A superposição permite que os qubits existam em múltiplos estados simultaneamente, o que aumenta exponencialmente o espaço de estado que pode ser explorado durante os cálculos. O emaranhamento cria uma ligação entre qubits, de modo que o estado de um qubit pode instantaneamente influenciar o estado de outro, independentemente da distância entre eles. Juntas, essas propriedades possibilitam a execução de operações em uma escala muito maior do que seria possível em sistemas clássicos.

Complementando a superposição e o emaranhamento, a interferência quântica é outro princípio crucial que permite manipular as probabilidades dos estados dos qubits. Por meio da interferência, é possível reforçar as soluções corretas de um problema enquanto cancela as incorretas. Essa propriedade é explorada em algoritmos quânticos para aumentar a eficiência na resolução de problemas complexos, como a busca em grandes bases de dados ou a simulação de sistemas moleculares. Assim, a interferência atua em conjunto com a superposição e o emaranhamento para amplificar o poder de computação dos qubits.

Para implementar esses conceitos, transformações condicionais e portas quânticas são fundamentais na manipulação precisa dos qubits. Portas como CNOT, Toffoli e Fredkin permitem realizar operações complexas, controlando o estado de um qubit com base no estado de outros, de forma semelhante às portas lógicas em circuitos digitais clássicos. A porta CNOT é essencial para criar o entrelaçamento de qubits, onde o estado de um qubit depende do estado de outro, mesmo que estejam separados por grandes distâncias. A porta Toffoli, por sua vez, pode realizar qualquer operação lógica necessária, enquanto a porta Fredkin oferece flexibilidade adicional ao permitir a troca dos estados dos qubits. Essas portas são a base para implementar a interferência e o emaranhamento em algoritmos práticos.

No entanto, a física quântica impõe restrições aos tipos de transformações que podem ser feitas. Todas as transformações de estado quântico, e portanto todos os portões quânticos e computações quânticas, devem ser reversíveis. Isso significa que cada transformação pode ser desfeita, permitindo uma manipulação mais eficiente dos dados. Na computação clássica, muitas operações são irreversíveis, levando à perda de informações e à necessidade de operações adicionais para gerenciar esses dados. A reversibilidade das operações quânticas evita esses problemas, resultando em um processamento mais eficiente e sustentável. Dessa forma, a reversibilidade não apenas otimiza a eficiência do processamento de informações, mas também aproveita ao máximo as propriedades fundamentais dos qubits.


\subsection{Criptografia quântica}

A criptografia quântica nos últimos anos tem sido alvo de muita pesquisa, tendo em vista que a tecnologia quântica representa uma ameaça significativa à criptografia tradicional. Os computadores quânticos possuem um poder de processamento que realizam operações em pararelo e exploram os princípios da superposição e emaranhamento, e isso poderá comprometer a segurança dos dados nas comunicações digitais. A criptografia tradicional utiliza como princípio algoritmos matemáticos, {já a criptografia quântica utiliza propriedades quânticas da luz para realizar tarefas criptográficas \cite{mitra2017quantum}. Essa ideia, que  inicialmente foi apresentada por Stephen Wiesner na década de 1970, é baseada no princípio da incerteza de Heisenberg \cite{heisenberg1927anschaulichen}. A criptografia quântica aproveita propriedades como o comportamento duplo da luz (onda e partícula) e a interdependência instantânea entre elétrons de um par. Enquanto a criptografia clássica é segura enquanto o algoritmo de encriptação for seguro, mas é vulnerável a engenharia reversa e \textit{hacking}, a criptografia moderna busca melhorar essas limitações usando chaves mais complexas e computação matemática avançada. Algoritmos de computação quântica, como o de Shor, podem quebrar muitos dos algoritmos criptográficos atuais em tempo significativamente menor. A criptografia quântica oferece segurança incondicional, mesmo contra adversários quânticos, e não depende da complexidade matemática, mas de princípios físicos quânticos.\cite{mitra2017quantum}.
	
	Conforme citado por Cameron R. Argetsinger a computação quântica pode ser o que vai salvar o mundo como o que vai acabar com o planeta, tudo depende de quem está manipulando essa tecnologia. Grupos criminosos já estão trabalhando nisso. O conceito de ``\textit{store now, decrypt later}'' (também referido como ``\textit{harvest now, decrypt later}'', ou ``\textit{steal now, decrypt later}'') refere-se à prática de interceptar e armazenar dados criptografados atualmente, com a expectativa de que avanços futuros na tecnologia de computação quântica permitirão decifrar esses dados \cite{argetsinger2024promise}.
	
	No contexto desta proposta de TCC, é de fundamental importância a tese de doutorado de \citeauthor{giron2023}.A tese discute como a segurança das conexões TLS, que utilizam o protocolo ACME (\textit{Automated Certificate Management Environment}), pode se tornar vulnerável à medida que computadores quânticos desenvolvem a capacidade de quebrar os algoritmos criptográficos atualmente em uso. Para enfrentar esse desafio, a criptografia pós-quântica (CPQ) é apresentada como uma solução viável, baseada em problemas matemáticos que são considerados resistentes tanto para computadores clássicos quanto para quânticos \cite{giron2023}. Este trabalho fornece uma base para explorar soluções de segurança que protejam a confidencialidade dos dados em um cenário de computação quântica.
	
	A criptografia pós-quântica (CPQ), portanto, é uma resposta à ameaça que a computação quântica representa para os métodos criptográficos atuais. Conforme abordado na pesquisa de \citeauthor{giron2023}, a padronização de algoritmos de CPQ pelo NIST, como Dilithium, Falcon e SPHINCS+, busca garantir a segurança mesmo na era da computação quântica. Além disso, a transição para CPQ é desafiadora e complexa devido a questões de desempenho das aplicações e protocolos de rede, e a confiança nos novos algoritmos pós-quânticos, por esse motivo, Giron sugere a CPQ híbrida que fará uma transição mais suave à criptografia quântica.
	
	No estudo realizado por \citeauthor{augusto2022}, a computação quântica não só transforma os paradigmas da segurança da informação, mas também impõe a necessidade de desenvolvimento contínuo de novos algoritmos que possam resistir aos ataques quânticos. A maior parte dos algoritmos de criptografia tradicionais é baseada na segurança matemática e que computadores convencionais não conseguiriam resolver. Um desses algoritmos é o RSA, sua base é cima de fatoração de números primos. Sua complexidade torna a quebra desse algoritmo altamente improvável na computação convencional, mas a estimativa é que um processador quântico com a capacidade de 6000 qubit levaria apenas 2 semanas para quebrá-lo \cite{augusto2022}.
	


%Uma forma de tratar o referencial teórico é definir como título de capítulo o assunto macro e relevante relacionado ao trabalho e o texto é dividido em subtítulos (seções e subseções), conforme necessário. Essa forma é preferida por deixar explícito o assunto a ser tratado e que o mesmo é a fundamentação do trabalho \footnote{Teste de nota de rodapé 3.}. 
%
%Outra forma de tratar esse capítulo é denominá-lo referencial teórico e dividi-lo em seções e subseções ou com um único texto os assuntos que fornecem o suporte teórico para o trabalho. Essa forma pode ser utilizada quando assuntos distintos fundamentam o trabalho e é difícil incluí-los sob uma mesma denominação de capítulo \footnote{Teste de nota de rodapé 4.}.
%
%O embasamento teórico se refere ao(s) assunto(s) principal(is) relacionado(s) ao objeto de pesquisa para o qual o trabalho traz alguma contribuição ou que é utilizado como referência conceitual para o desenvolvimento do proposto no trabalho. O assunto pode fornecer a fundamentação (suporte teórico) para a ideia do sistema, para definir claramente o problema, para explicitar a solução, para a forma de resolução; referir-se aos conceitos e teorias relacionados ao sistema desenvolvido, sobre tecnologias e metodologias específicas utilizadas na definição do sistema e na sua implementação.
%
%Exemplos:
%
%Conceitos da orientação a objetos fazem parte do referencial teórico se o uso intensivo da orientação a objetos é o principal embasamento do trabalho; ou se a principal contribuição do trabalho está relacionada à orientação a objetos, seja em termos de agregar conhecimento nessa área ou à forma de usar os seus conceitos.
%
%Sistemas distribuídos pode ser o assunto do embasamento teórico se o resultado do trabalho for um sistema distribuído. O mesmo pode ocorrer com sistemas cliente servidor, sistemas de informações gerenciais, de apoio à decisão, para web e etc.
%
%Se o desenvolvimento de um sistema para biometria for o objeto do trabalho, o referencial teórico se refere aos conceitos principais de biometria, aplicabilidade, exemplos de sistemas existentes, o que esses sistemas tratam, como eles são, etc.
%
%Se um sistema web para portadores de necessidades especiais for o resultado do trabalho, o referencial teórico refere-se as quais e como são essas necessidades, outros sistemas existentes na área, como os sistemas lidam com essas necessidades e os principais conceitos por eles considerados.
%
%O embasamento teórico pode conter os trabalhos relacionados, desde que seja relevante para o desenvolvimento do trabalho. Esse item deve ser elaborado especialmente quando se trata do desenvolvimento de algo muito específico, havendo a necessidade de um estudo comparativo. Nesse caso pode-se inserir claramente o trabalho de pesquisa no contexto dos demais autores, no sentido da contribuição da proposta na área de pesquisa em que o mesmo se insere e em relação ao que já tem pesquisado na área. 
%
%\caixa{Atenção}{Converse com o seu orientador para ver quais seções/conteúdos devem ter neste capítulo...}
%
%\section{Observações sobre a citações}\label{sec:formatacaoTexto}
%
%O texto em si é dividido em títulos e subtítulos, se necessário. 
%
%O espaçamento entre linhas é de 1,5. Os títulos das seções primárias e das demais subseções devem ser separados do texto que os precede ou que os sucede por uma linha em branco. As seções primárias devem iniciar em páginas distintas.
%
%Com relação à paginação, todas as folhas do trabalho, a partir da folha de rosto, devem ser contadas sequencialmente, mas não numeradas. A numeração deve ser colocada a partir da primeira folha da parte textual (introdução), em algarismos arábicos, no canto superior direito da folha.
%
%\caixa{Observação}{Se você estiver utilizando \latex, não é necessário se preocupar com formatação.}
%
%As próximas seções comentam a respeito de citações.
%
%\subsection{Citações}\label{subsec:citacoes}
%
%\textbf{Citação direta:} É quando o texto utilizado é transcrito com as próprias palavras do autor. Quando curtas (até três linhas) a transcrição literal virá entre “aspas” e a referência pode ser incluída no texto junto à sentença ou frase, ou ainda ser colocada entre parênteses. Quando inclusa no texto, deve-se usar letras maiúsculas e minúsculas, com indicação da data e demais informações entre parênteses.
%
%Exemplo de citação direta curta com autor incluso no texto: Segundo \citeonline[p. 107]{Pressman2009} o valor da informação está “diretamente ligado à maneira como ela ajuda os tomadores de decisões a atingirem as metas da organização”. Exemplo de citação direta curta com autor não incluso no texto: O autor lembra, contudo, a análise precursora de \citeonline{Pressman2009} sobre alguns aspectos limitantes das competências, ou aptidões, essenciais, que as transformam em “limitações estratégicas” \cite{Pressman2009}.
%
%As transcrições com mais de três linhas (citações diretas longas) aparecem recuadas em 4 cm, a partir da margem esquerda, em espaço simples, tamanho 10, e a indicação da fonte é apresentada entre parênteses. 
%
%\begin{citacao}
%Na nova sociedade, chamada de capitalista: O recurso econômico básico – ‘os meios de produção’, para usar uma expressão dos economistas – não é mais o capital, nem os recursos naturais (a ‘terra’ dos economistas), nem a ‘mão-de-obra’. Ele será o conhecimento. As atividades centrais de criação de riqueza não serão nem a alocação de capital para usos produtivos, nem a ‘mão-de-obra’ – os dois pólos da teoria econômica dos séculos dezenove e vinte, quer ela seja clássica, marxista, keynesiana ou neoclássica. Hoje o valor é criado pela ‘produtividade’ e pela ‘inovação’, que são aplicações do conhecimento ao trabalho. Os principais grupos sociais da sociedade do conhecimento serão os ‘trabalhadores do conhecimento’ – executivos que sabem como alocar conhecimento para usos produtivos. \cite[p. 48]{Pressman2009}.
%\end{citacao}
%
%\textbf{Citação indireta:} É a reprodução de ideias do autor. É uma citação livre, usando as palavras de quem está escrevendo para dizer o mesmo que o autor disse no texto. Contudo, a ideia expressa continua sendo de autoria do autor consultado, por isso é necessário citar a fonte: dar crédito ao autor da ideia. Exemplo de citação indireta: O valor da informação está relacionado com o poder de ajuda aos tomadores de decisões a atingirem os objetivos da empresa\cite{Pressman2009}. Outra forma de citação indireta: \citeonline{Pressman2009} destacam ser fundamental a gestão de dados nas organizações, pois isso garantirá o funcionamento normal dos sistemas de informação, uma vez que, sem a capacidade de seu processamento, haveria problemas para a empresa executar suas atividades efetivamente.
%
%Citações de obras que contenham até três autores, devem apresentar os sobrenomes destes separados por ponto e vírgula, como no exemplo: \cite[p. 2]{Pinto2000}. E para obras que contenham mais de três autores indica-se citar apenas o nome do primeiro autor, seguido da expressão abreviada \textit{et al.}, como no exemplo: \cite{Guimaraes2003}.
%
%\subsection{Ilustrações, quadros e tabelas}\label{subsec:ilustracoes}
%
%As ilustrações, quadros e tabelas devem aparecer no texto, segundo a NBR14724:2011, de forma padronizada.
%
%Qualquer que seja o tipo de ilustração, sua identificação aparece na parte superior, precedida da palavra designativa (desenho, esquema, fluxograma, fotografia, gráfico, mapa, organograma, planta, quadro, retrato, figura, imagem, entre outros), seguida de seu número de ordem de ocorrência no texto, em algarismos arábicos, travessão e do respectivo título. Após a ilustração, na parte inferior, indicar a fonte consultada (elemento obrigatório, mesmo que seja produção do próprio autor), legenda, notas e outras informações necessárias à sua compreensão (se houver). A ilustração deve ser citada no texto e inserida o mais próximo possível do trecho a que se refere.
%
%A fonte, ou seja, a indicação do autor da ilustração ou da publicação de onde ela foi retirada deve aparecer na parte inferior. Exemplo:
%
%Fonte: \citeonline{Coulouris2013}. 			- quando utilizado o item original
%
%Fonte: Adaptado de \citeonline{Coulouris2013}.	- quando o item original foi alterado
%
%Para facilitar a inclusão de fontes, o \textit{template} em LaTeX da \gls{utfpr}, possui o comando \texttt{$\backslash$fonte\{\}}. Se este comando for deixado em branco (\texttt{$\backslash$fonte\{\}}),  ele preencherá automaticamente a fonte com o texto  ``Fonte: Autoria própria (ANO)'', sendo ANO substituído pelo ano atual. Já se o comando \texttt{$\backslash$fonte\{\}} tiver algum conteúdo (não estiver em branco), tal conteúdo será inserido na legenda da fonte e esse conteúdo pode ser uma citação. Por exemplo, o comando \texttt{$\backslash$fonte\{$\backslash$citeonline\{Coulouris2013\}\}} gerará o texto ``Fonte: \citeonline{Coulouris2013}.''. Atenção, não é necessário incluir o ponto final (``.''), no texto do comando \texttt{$\backslash$fonte\{\}}, pois isso é feito automaticamente.  
%
%A figura também deve ser citada no texto. Primeira opção, como pode ser observado na \autoref{fig:exemplo1}. Segunda opção, como pode ser observado na Figura \ref{fig:exemplo1}.
%
%\begin{figure}[htb]%% Ambiente figure
%    %\captionsetup{width=0.55\textwidth}%% Largura da legenda
%    \caption{Exemplo de figura criada a partir de um arquivo}%% Legenda
%    \label{fig:exemplo1}%% Rótulo
%    \includegraphics[scale=0.4]{cs2}%% Dimensões e localização
%    \fonte{Adaptado de \citeonline{Coulouris2013}}%% Fonte
%\end{figure}
%
%Utilizando o pacote \textit{subfig} é possível adicionar figuras lado a lado, como pode ser observado na \autoref{fig:exemplo2}.
%
%\begin{figure}[htb]
%    \caption{Telas de cadastro de Paciente: (a) Cadastro Paciente, (b) Cadastro Paciente 2} 
%	\label{fig:exemplo2}
%	\centering
%	\subfloat[Cadastro Paciente]{
%		\includegraphics[scale=0.7]{cadastro-paciente}
%	}\hspace{0.15cm} 
%	\subfloat[Cadastro Paciente 2]{
%		\includegraphics[scale=0.7]{cadastro-paciente}
%	}
%	
%	\fonte{}
%\end{figure}
%
%Este modelo vem com o ambiente \texttt{quadro} e impressão de Lista de quadros 
%configurados por padrão.  Este parágrafo apresenta como referenciar o quadro no texto, requisito obrigatório da ABNT. Primeira opção, utilizando \texttt{autoref}: Ver o \autoref{quad:exemplo1}. Segunda opção, utilizando  \texttt{ref}: Ver o Quadro \ref{quad:exemplo1}.
%
%\begin{tabframed}[htb]%% Ambiente tabframed
%%\captionsetup{width=0.5\textwidth}%% Largura da legenda
%\caption{Materiais utilizados no desenvolvimento do sistema}%% Legenda
%\label{quad:exemplo1}%% Rótulo
%\renewcommand{\arraystretch}{1.5}
%\begin{tabular}{|l|l|l|l|l}
%\cline{1-4}
%\textbf{Ferramenta/Tecnologia} & \textbf{Versão} & \textbf{Disponível em} & \textbf{Finalidade} \\ \cline{1-4}
% Teste & 1.0  & https:/teste.org & Biblioteca de Teste & \\ \cline{1-4}  
% Teste & 1.0  & https:/teste.org & Biblioteca de Teste & \\ \cline{1-4}
% Teste & 1.0  & https:/teste.org & Biblioteca de Teste & \\ \cline{1-4}
% Teste & 1.0  & https:/teste.org & Biblioteca de Teste & \\ \cline{1-4}
%\end{tabular}
%\fonte{}%% Fonte
%\end{tabframed}
%
%
%Também é possível citar tabelas no texto. Primeira opção, utilizando \texttt{autoref}: Ver o \autoref{tab:exemplo1}. Segunda opção, utilizando  \texttt{ref}: Ver a Tabela \ref{tab:exemplo1}.
%
%\begin{table}[htb]
%% Luiz - O texto do caption da tabela/quadro deve ser do tamanho da tabela, então utilize a linha a seguir para conseguir esse efeito
%\captionsetup{width=0.33\textwidth}
%\centering
%\caption{\label{tab:exemplo1}Exemplo de tabela com uma legenda contendo um texto longo}
%\begin{tabular}{cccc}
%	\hline
%	\textbf{Pessoa} & \textbf{Idade} & \textbf{Peso} & \textbf{Altura} \\ \hline
%	Marcos & 26    & 68   & 178    \\ 
%	Ivone  & 22    & 57   & 162    \\ 
%	...    & ...   & ...  & ...    \\ 
%	Sueli  & 40    & 65   & 153    \\ \hline
%\end{tabular}
%\fonte{}
%\end{table}
%
%A \autoref{tab:exemplo2} também pode ser citada no texto.
%
%\begin{table}[htb]%% Ambiente table
%\caption{Segundo exemplo de tabela com uma legenda contendo um texto muito longo que pode ocupar mais de uma linha}%% Legenda
%\label{tab:exemplo2}%% Rótulo
%\begin{tabularx}{\textwidth}{@{\extracolsep{\fill}}llll}%% Ambiente tabularx
%\toprule
%$\bsym{L}$ & $\bsym{L^2}$ & $\bsym{L^3}$ & $\bsym{L^4}$ \\
%\SI{}{[m]} & \SI{}{[m^2]} & \SI{}{[m^3]} & \SI{}{[m^4]} \\ \midrule
%1          & 1            & 1            & 1            \\
%2          & 4            & 8            & 16           \\
%3          & 9            & 27           & 81           \\
%4          & 16           & 64           & 256          \\
%5          & 25           & 125          & 625          \\ \bottomrule
%\end{tabularx}
%\fonte{}%% Fonte
%\end{table}
%
%A \autoref{tab:exemplo3} é um exemplo de tabela que ocupa mais de uma página e que foi construída pelo \gls{latex}\index{LaTeX@\latex} utilizando o pacote \texttt{longtable}.
%
%\begin{longtable}{@{\extracolsep{\fill}}lll}%% Ambiente longtable
%\caption{Possíveis tríplices para grade altamente variável\label{tab:exemplo3}} \\%% Legenda e rótulo
%\toprule
%\textbf{Tempo (s)} & \textbf{Tríplice escolhida} & \textbf{Outras possíveis tríplices} \\
%\midrule
%\endfirsthead%% Encerra cabeçalho da primeira página
%\caption[]{Possíveis tríplices para grade altamente variável} \\%% Legenda
%\multicolumn{3}{r}{\textbf{(continuação)}} \\
%\toprule
%\textbf{Tempo (s)} & \textbf{Tríplice escolhida} & \textbf{Outras possíveis tríplices} \\
%\midrule
%\endhead%% Encerra cabeçalho das demais páginas
%\midrule
%\multicolumn{3}{r}{\textbf{(continua)}} \\
%\endfoot%% Encerra rodapé das demais páginas
%\bottomrule
%\\[-0.5\linha]
%\caption*{\nomefonte: Adaptado de \citet{Smallen2014}} \\
%\endlastfoot%% Encerra rodapé da última página
%0      & (1, 11, 13725) & (1, 12, 10980), (1, 13, 8235), (2, 2, 0), (3, 1, 0) \\
%2745   & (1, 12, 10980) & (1, 13, 8235), (2, 2, 0), (2, 3, 0), (3, 1, 0)      \\
%5490   & (1, 12, 13725) & (2, 2, 2745), (2, 3, 0), (3, 1, 0)                  \\
%8235   & (1, 12, 16470) & (1, 13, 13725), (2, 2, 2745), (2, 3, 0), (3, 1, 0)  \\
%10980  & (1, 12, 16470) & (1, 13, 13725), (2, 2, 2745), (2, 3, 0), (3, 1, 0)  \\
%13725  & (1, 12, 16470) & (1, 13, 13725), (2, 2, 2745), (2, 3, 0), (3, 1, 0)  \\
%16470  & (1, 13, 16470) & (2, 2, 2745), (2, 3, 0), (3, 1, 0)                  \\
%19215  & (1, 12, 16470) & (1, 13, 13725), (2, 2, 2745), (2, 3, 0), (3, 1, 0)  \\
%21960  & (1, 12, 16470) & (1, 13, 13725), (2, 2, 2745), (2, 3, 0), (3, 1, 0)  \\
%24705  & (1, 12, 16470) & (1, 13, 13725), (2, 2, 2745), (2, 3, 0), (3, 1, 0)  \\
%27450  & (1, 12, 16470) & (1, 13, 13725), (2, 2, 2745), (2, 3, 0), (3, 1, 0)  \\
%30195  & (2, 2, 2745)   & (2, 3, 0), (3, 1, 0)                                \\
%32940  & (1, 13, 16470) & (2, 2, 2745), (2, 3, 0), (3, 1, 0)                  \\
%35685  & (1, 13, 13725) & (2, 2, 2745), (2, 3, 0), (3, 1, 0)                  \\
%38430  & (1, 13, 10980) & (2, 2, 2745), (2, 3, 0), (3, 1, 0)                  \\
%41175  & (1, 12, 13725) & (1, 13, 10980), (2, 2, 2745), (2, 3, 0), (3, 1, 0)  \\
%43920  & (1, 13, 10980) & (2, 2, 2745), (2, 3, 0), (3, 1, 0)                  \\
%46665  & (2, 2, 2745)   & (2, 3, 0), (3, 1, 0)                                \\
%49410  & (2, 2, 2745)   & (2, 3, 0), (3, 1, 0)                                \\
%52155  & (1, 12, 16470) & (1, 13, 13725), (2, 2, 2745), (2, 3, 0), (3, 1, 0)  \\
%54900  & (1, 13, 13725) & (2, 2, 2745), (2, 3, 0), (3, 1, 0)                  \\
%57645  & (1, 13, 13725) & (2, 2, 2745), (2, 3, 0), (3, 1, 0)                  \\
%60390  & (1, 12, 13725) & (2, 2, 2745), (2, 3, 0), (3, 1, 0)                  \\
%63135  & (1, 13, 16470) & (2, 2, 2745), (2, 3, 0), (3, 1, 0)                  \\
%65880  & (1, 13, 16470) & (2, 2, 2745), (2, 3, 0), (3, 1, 0)                  \\
%68625  & (2, 2, 2745)   & (2, 3, 0), (3, 1, 0)                                \\
%71370  & (1, 13, 13725) & (2, 2, 2745), (2, 3, 0), (3, 1, 0)                  \\
%74115  & (1, 12, 13725) & (2, 2, 2745), (2, 3, 0), (3, 1, 0)                  \\
%76860  & (1, 13, 13725) & (2, 2, 2745), (2, 3, 0), (3, 1, 0)                  \\
%79605  & (1, 13, 13725) & (2, 2, 2745), (2, 3, 0), (3, 1, 0)                  \\
%82350  & (1, 12, 13725) & (2, 2, 2745), (2, 3, 0), (3, 1, 0)                  \\
%85095  & (1, 12, 13725) & (1, 13, 10980), (2, 2, 2745), (2, 3, 0), (3, 1, 0)  \\
%87840  & (1, 13, 16470) & (2, 2, 2745), (2, 3, 0), (3, 1, 0)                  \\
%90585  & (1, 13, 16470) & (2, 2, 2745), (2, 3, 0), (3, 1, 0)                  \\
%93330  & (1, 13, 13725) & (2, 2, 2745), (2, 3, 0), (3, 1, 0)                  \\
%96075  & (1, 13, 16470) & (2, 2, 2745), (2, 3, 0), (3, 1, 0)                  \\
%98820  & (1, 13, 16470) & (2, 2, 2745), (2, 3, 0), (3, 1, 0)                  \\
%101565 & (1, 13, 13725) & (2, 2, 2745), (2, 3, 0), (3, 1, 0)                  \\
%104310 & (1, 13, 16470) & (2, 2, 2745), (2, 3, 0), (3, 1, 0)                  \\
%107055 & (1, 13, 13725) & (2, 2, 2745), (2, 3, 0), (3, 1, 0)                  \\
%109800 & (1, 13, 13725) & (2, 2, 2745), (2, 3, 0), (3, 1, 0)                  \\
%112545 & (1, 12, 16470) & (1, 13, 13725), (2, 2, 2745), (2, 3, 0), (3, 1, 0)  \\
%115290 & (1, 13, 16470) & (2, 2, 2745), (2, 3, 0), (3, 1, 0)                  \\
%118035 & (1, 13, 13725) & (2, 2, 2745), (2, 3, 0), (3, 1, 0)                  \\
%120780 & (1, 13, 16470) & (2, 2, 2745), (2, 3, 0), (3, 1, 0)                  \\
%123525 & (1, 13, 13725) & (2, 2, 2745), (2, 3, 0), (3, 1, 0)                  \\
%126270 & (1, 12, 16470) & (1, 13, 13725), (2, 2, 2745), (2, 3, 0), (3, 1, 0)  \\
%129015 & (2, 2, 2745)   & (2, 3, 0), (3, 1, 0)                                \\
%131760 & (2, 2, 2745)   & (2, 3, 0), (3, 1, 0)                                \\
%134505 & (1, 13, 16470) & (2, 2, 2745), (2, 3, 0), (3, 1, 0)                  \\
%137250 & (1, 13, 13725) & (2, 2, 2745), (2, 3, 0), (3, 1, 0)                  \\
%139995 & (2, 2, 2745)   & (2, 3, 0), (3, 1, 0)                                \\
%142740 & (2, 2, 2745)   & (2, 3, 0), (3, 1, 0)                                \\
%145485 & (1, 12, 16470) & (1, 13, 13725), (2, 2, 2745), (2, 3, 0), (3, 1, 0)  \\
%148230 & (2, 2, 2745)   & (2, 3, 0), (3, 1, 0)                                \\
%150975 & (1, 13, 16470) & (2, 2, 2745), (2, 3, 0), (3, 1, 0)                  \\
%153720 & (1, 12, 13725) & (2, 2, 2745), (2, 3, 0), (3, 1, 0)                  \\
%156465 & (1, 13, 13725) & (2, 2, 2745), (2, 3, 0), (3, 1, 0)                  \\
%159210 & (1, 13, 13725) & (2, 2, 2745), (2, 3, 0), (3, 1, 0)                  \\
%161955 & (1, 13, 16470) & (2, 2, 2745), (2, 3, 0), (3, 1, 0)                  \\
%164700 & (1, 13, 13725) & (2, 2, 2745), (2, 3, 0), (3, 1, 0)                  \\
%\end{longtable}
%
%
%\subsection{Códigos fonte e algoritmos}\label{subsec:algoritimos}
%
%Os algoritmos podem ser utilizados para explicar uma determinada rotina desenvolvida. Conforme pode ser observado no \autoref{alg:exemplo1}.
%
%\begin{algorithm}[htb]%% Ambiente algorithm
%\caption{Algoritmo de exemplo}%% Legenda
%\label{alg:exemplo1}%% Rótulo
%\hrule
%\begin{algorithmic}[1]%% Ambiente algorithmic
%\ENSURE $A, B$
%\STATE $C = A + B$
%\IF{$C < 10$}
%\STATE $C = 2 \ C$
%\ELSE
%\STATE $C = 0,5 \ C$
%\ENDIF
%\PRINT $A, B, C$
%\end{algorithmic}
%\hrule
%\fonte{}%% Fonte
%\end{algorithm}
%
%\lipsum[1]
%
%\lipsum[1]
%
%Na \autoref{code:exemplo1} pode ser visualizado um exemplo de código fonte.
%
%\begin{sourcecode}[htb]
%\caption{\label{code:exemplo1}Exemplo de código}
%\begin{lstlisting}[frame=single, language=Java]
%@Entity
%public class Foo {
% 
%    @Id
%    @GeneratedValue(strategy = GenerationType.IDENTITY)
%    private Long id;
% 
%    private String name;
%    // constructor, getters and setters
%}
%\end{lstlisting}
%\fonte{}
%\end{sourcecode}

