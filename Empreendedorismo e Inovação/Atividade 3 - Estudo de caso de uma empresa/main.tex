\documentclass{article}
\usepackage{amsmath}
\usepackage{graphicx}
\usepackage[utf8]{inputenc}
\usepackage[portuguese]{babel}

\title{Estudo de Caso: Ciclo de Vida Organizacional da Empresa X}
\date{}

\begin{document}
	
	\maketitle
	
	\section*{Alunos: }
	João Pedro Rodrigues Leite RA: A2487055 \\
	Jacson Siqueira César RA: A2487012 \\
	Lucas José da Costa RA: A2487071 \\
	Ulisses Meotti RA: A2460882
	
	\section*{Atividade}
	1. Faça um estudo de caso em uma organização, identifique em qual fase do ciclo de vida ela se encontra, considerando os conceitos apresentados neste capítulo. Além disso, aponte os pontos fortes e fracos da organização (fatores internos), também elenque as suas oportunidades e ameaças (fatores externos).
	
	\section{Introdução}
	A empresa X é uma startup de tecnologia que desenvolve soluções de software para pequenas e médias empresas. Com 5 anos de mercado, já passou por várias etapas e enfrenta os desafios do crescimento. Vamos identificar em que fase do ciclo de vida organizacional ela está e fazer uma análise SWOT para entender melhor sua situação atual.
	
	\section{Fase do Ciclo de Vida}
	Hoje, a X está na fase de \textbf{Concepção/existência; Sobrevivência; Lucratividade/estabilização; Lucratividade/crescimento; Decolagem; Maturidade}(\textit{escolher um desses}). Ela já alcançou um nível de estabilidade financeira X está crescendo X. Nesse momento, os fundadores estão delegando mais responsabilidades para uma equipe de gestores, além de buscar recursos financeiros para continuar expandindo.
	
	\section{Análise SWOT}
	A análise SWOT (Pontos Fortes, Fracos, Oportunidades e Ameaças) nos ajuda a entender melhor os desafios e vantagens da TechNova.
	
	\subsection{Pontos Fortes (Fatores Internos)}
	\begin{itemize}
		\item \textbf{Inovação:} A empresa é muito criativa, sempre lançando produtos inovadores no mercado.
		\item \textbf{Delegação:} Os fundadores estão conseguindo passar as responsabilidades gerenciais para outras pessoas, o que ajuda no crescimento.
		\item \textbf{Recursos Financeiros:} A x conseguiu bons investimentos e está com capital suficiente para expandir.
	\end{itemize}
	
	\subsection{Pontos Fracos (Fatores Internos)}
	\begin{itemize}
		\item \textbf{Dependência de Grandes Clientes:} A empresa depende muito de alguns clientes grandes, o que pode ser arriscado.
		\item \textbf{Sistemas Imaturos:} Alguns processos internos ainda precisam ser mais estruturados.
	\end{itemize}
	
	\subsection{Oportunidades (Fatores Externos)}
	\begin{itemize}
		\item \textbf{Expansão para Outros Países:} Há a chance de entrar no mercado internacional e crescer ainda mais.
		\item \textbf{Crescimento da Tecnologia:} A demanda por soluções tecnológicas está em alta, o que abre muitas portas.
	\end{itemize}
	
	\subsection{Ameaças (Fatores Externos)}
	\begin{itemize}
		\item \textbf{Concorrência Forte:} Muitas empresas novas e grandes estão competindo no mercado.
		\item \textbf{Mudanças Rápidas na Tecnologia:} A tecnologia muda muito rápido, e a empresa precisa se adaptar constantemente.
	\end{itemize}
	
	\section{Conclusão}
	A x está em um momento crucial. Ela está crescendo, mas precisa continuar delegando responsabilidades e melhorando seus processos internos. A empresa tem uma boa base de inovação e oportunidades no mercado, mas deve se cuidar para não ser prejudicada pela concorrência ou por sua dependência de grandes clientes.
	
\end{document}
