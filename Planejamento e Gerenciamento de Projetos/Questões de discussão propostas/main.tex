\documentclass{article}
\usepackage[utf8]{inputenc}
\usepackage{amsmath}

\title{Questões de discussão propostas}
\author{Aluno: João Pedro Rodrigues Leite \\ RA: A2487055}
\date{}

\begin{document}
	
	\maketitle

	\subsubsection*{1) Baseando-se no planejamento estratégico de uma empresa, crie um perfil para uma empresa fictícia, ou que já exista, e elabore sua missão, visão e valores.}
	
	\textbf{Empresa fictícia:} FlowersTech \\
	\textbf{Perfil:} Pioneirismo e Inovação \\
	\textbf{Missão:} Inovar na tecnologia, trazendo dispositivos inteligentes, tais como dispositivos para monitoramento de plantas em geral. \\
	\textbf{Visão:} Criar tecnologia inovadora que seja acessível a todos e que se adapte às necessidades de cada usuário. \\
	\textbf{Valores:} Qualidade, Inovação, Responsabilidade, Confiança.
	
	\subsubsection*{2) Explique o papel que os projetos desempenham no processo de planejamento estratégico}
	
	O papel que o projeto desempenha nesse processo de planejamento estratégico é focado no pioneirismo e inovação, trazendo tecnologia para dispositivos inteligentes que irão se adaptar de acordo com as necessidades do usuário.
	
	\subsubsection*{3) Via de regra, o portfólio de projetos é representado por projetos de conformidade, estratégicos e operacionais. Que efeito essa classificação pode ter na seleção de projetos (Positivo e/ou negativo: explique)}
	
	Acredito que podemos olhar os dois efeitos que podem acontecer.
	
	\begin{itemize}
		\item \textbf{Efeito Positivo:} Irá facilitar a alocação de recursos, priorizando projetos que estejam mais alinhados com os objetivos de longo prazo. Essa divisão possibilita uma boa análise dos impactos financeiros, operacionais e de conformidade. Desse modo garante que projetos estratégicos, que são fundamentais para o crescimento e a inovação, sejam priorizados.
		
		\item \textbf{Efeito Negativo:} Pode ocorrer uma sobrecarga de projetos operacionais ou de conformidade que desviem recursos de projetos estratégicos. Além disso, a rigidez na classificação pode dificultar a adaptação a mudanças rápidas no mercado, focando excessivamente em objetivos já estabelecidos e negligenciando inovações emergentes.
	\end{itemize}
	
	\subsubsection*{4) Por que a organização não deve se fiar apenas no ROI (ou critérios financeiros) para selecionar projetos?}
	
	
	A organização não deve se fiar apenas no ROI (ou critérios financeiros) para selecionar projetos porque isso pode levar à negligência de fatores críticos para o sucesso a longo prazo, como inovação, impacto social, sustentabilidade e alinhamento com os valores da empresa.
	
	
\end{document}
