\documentclass{article}
\usepackage[utf8]{inputenc}
\usepackage{amsmath}

\title{Questionário 1 - Aplicação Prática}
\author{Aluno: João Pedro Rodrigues Leite \\ RA: A2487055}
\date{}

\begin{document}
	
	\maketitle
	
	\section*{1) Escolha do título para um projeto}
	\textbf{Resposta:} Sistema de monitoramento de plantas com inteligência artificial
	
	\section*{2) Defina o seu projeto em termos de características a partir de exemplos}
	\begin{itemize}
		\item \textbf{Objetivo bem definido:} O projeto busca desenvolver um sistema de monitoramento inteligente para plantas, usando algoritmos de Intelgiência Artificial que analisam dados em tempo real, isso irá ajudar no cuidado com as plantas de forma automatizada.
		
		\item \textbf{Ciclo de vida definido:} O projeto começa com a criação dos sensores, passa pela fase de desenvolvimento do software, e termina com a implementação do sistema completo.
		
		\item \textbf{Envolve esforço multidisciplinar:} Esse projeto vai precisar de profissionais de diferentes áreas, como engenheiros para os sensores, programadores para a IA, e especialistas em plantas para ajustar o sistema às necessidades reais delas. Portanto será um trabalho colaborativo de várias equipes.
		
		\item \textbf{Resultado único:} Acredito que esse sistema é algo inovador dentro do contexto atual, pois vai reunir dados ambientais e tomar decisões automatizadas, algo que ainda não foi feito nessa área com essa abordagem específica.
		
		\item \textbf{Requisitos de tempo, custo e qualidade:} Acredito que para esse projeto seria possível entregá-lo em 6 meses, com um orçamento de R\$ 100.000.
	\end{itemize}

	
	\section*{3) Identifique 2 atividades para cada fase do ciclo de vida}
	\begin{itemize}
		\item \textbf{Inicialização:}
		\begin{itemize}
			\item Identificação das necessidades do cliente e definição de objetivos principais.
			\item Análise de viabilidade técnica e econômica do projeto.
		\end{itemize}
		\item \textbf{Planejamento:}
		\begin{itemize}
			\item Elaboração do cronograma detalhado do projeto.
			\item Definição dos recursos necessários (equipe, materiais, tecnologia).
		\end{itemize}
		\item \textbf{Execução:}
		\begin{itemize}
			\item Desenvolvimento do software de monitoramento.
			\item Desenvolvimento dos sensores e integração com o software de monitoramento.
			\item Testes iniciais do sistema.
		\end{itemize}
		\item \textbf{Monitoramento e Controle:}
		\begin{itemize}
			\item Acompanhamento do progresso do desenvolvimento do software e dos sensores.
			\item Avaliação da precisão dos dados coletados e ajustes no algoritmo de inteligência artificial.
		\end{itemize}
		\item \textbf{Encerramento:}
		\begin{itemize}
			\item Entrega final do sistema de monitoramento e treinamento da equipe usuária.
			\item Avaliação dos resultados.
		\end{itemize}
	\end{itemize}
	
	\section*{4) Sobre o Triangulo de Restrições para projeto, apresente uma	explicação de suas relações a partir de exemplos extraídos do seu projeto.}
	O Triângulo de Restrições é composto por três elementos principais: escopo, prazo e custo. No contexto do meu projeto, essas restrições estão interligadas da seguinte maneira:
	
	\begin{itemize}
		\item \textbf{Escopo:} O sistema de monitoramento precisa incluir sensores para coletar dados sobre temperatura, umidade e luminosidade, além de um software de análise que interpreta esses dados e um módulo para ações automatizadas (como ajustes de irrigação). Se o cliente quiser adicionar novas funcionalidades, como por exemplo um sensor que ative a pulverização de água automaticamente e envie notificações para o celular, isso vai aumentar tanto a complexidade quanto o tempo de desenvolvimento. O escopo expandido exigirá mais tempo para projetar e testar essas novas funcionalidades, além de aumentar o custo dos equipamentos e do desenvolvimento do software.
		
		\item \textbf{Prazo:} O projeto foi planejado para ser concluído em 6 meses. Se houver atrasos, por exemplo, na entrega dos sensores ou dificuldades no desenvolvimento do software, será necessário rever o cronograma. Para não estourar o prazo, seria possível contratar mais pessoal, o que aumentaria os custos, ou ajustar o escopo, talvez removendo funcionalidades menos prioritárias. Caso o cliente decida que quer uma nova funcionalidade, como a capacidade de contar quantas larvas estão presentes nas plantas e notificá-lo no celular, o prazo precisaria ser ajustado, pois essa seria uma nova funcionalidade complexa que requer novos sensores e algoritmos de análise de imagem.
		
		\item \textbf{Custo:} Acredito que o orçamento de R\$ 100.000 deve cobrir todos os gastos relacionados ao projeto, incluindo equipamentos, desenvolvimento de software e mão de obra. No entanto, se o cliente solicitar uma funcionalidade adicional, como o sensor para pulverização automática com notificação no celular ou o sistema de contagem de larvas, isso pode aumentar significativamente o custo. O hardware extra, como sensores específicos, e o desenvolvimento de novas funcionalidades no software, terão impacto no orçamento. Além disso, com o aumento do escopo, o prazo também poderá ser estendido, o que, por sua vez, eleva os custos gerais do projeto.
	\end{itemize}

	
\end{document}
